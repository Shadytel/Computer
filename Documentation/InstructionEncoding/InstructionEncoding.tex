\title{Instruction Encoding}
\date{\today}

\documentclass[10pt]{article}

\usepackage{amssymb}
\usepackage{amsmath}

\begin{document}
\maketitle

\section{Instructions}

The computer has 12-bit words.  Every instruction is exactly one word in size, although trickery can be used to write instructions which seem to be two words in length.

The table below summarizes the encoding of instructions on the machine.

\begin{tabular}{|c|c|c|c|c|c||c|c|c|c|c|c||l|}
\hline
\multicolumn{6}{|c||}{Opcode} & \multicolumn{6}{c||}{Operands} & Description \\
\hline
\hline
0 & 0 & C & \multicolumn{3}{c||}{PPP} & \multicolumn{3}{c|}{AAA} & \multicolumn{3}{c||}{BBB} & Math\\
\hline
0 & 1 & \multicolumn{4}{c||}{PPPP} & \multicolumn{3}{c|}{AAA} & \multicolumn{3}{c||}{BBB} & Logic \\
\hline
1 & 0 & \multicolumn{2}{c|}{RR} & \multicolumn{2}{c||}{QQ} & \multicolumn{3}{c|}{AAA} & \multicolumn{3}{c||}{BBB} & Load \\
\hline
1 & 1 & 0 & 0 & 1 & 0 & \multicolumn{3}{c|}{AAA} & \multicolumn{3}{c||}{BBB} & Sets ``compare'' on equality \\
\hline
1 & 1 & 0 & 1 & 0 & \multicolumn{4}{c|}{KKKK} & \multicolumn{3}{c||}{AAA} & Right rotate \\
\hline
1 & 1 & 0 & 1 & 1 & C & \multicolumn{3}{c|}{PPP} & \multicolumn{3}{c||}{BBB} & Immediate math and logic \\
\hline
\end{tabular}

In this table:
\begin{itemize}
\item ``AAA'' is ``source register;''
\item ``BBB'' is ``destination register;''
\item ``PPP(P)'' is ``function selection;''
\item ``KKKK'' is ``shift positions;''
\item ``RR'' is ``(source) address mode;''
\item ``QQ'' is ``(destination) address mode.''
\end{itemize}

Both ``RR'' and ``QQ'' are encoded as:

\begin{tabular}{|c|l|c|}
\hline
Encoding & Description & Code \\
\hline
\hline
0 & register direct & $r$ \\
\hline
1 & register dereference & $(r)$ \\
\hline
2 & register dereference with post-increment & $(r++)$ \\
\hline
3 & register dereference with pre-decrement & $(--r)$ \\
\hline
\end{tabular}


\section{Math}

The table below summarizes the math functions.

\begin{tabular}{|c||l|l|}
\hline
Encoding (PPP) & Instruction & Behavior \\
\hline
\hline
0 & add & $A+B \to B$ \\
\hline
1 & sub & $A-B \to B$ \\
\hline
2 & nsub & $B-A \to B$ \\
\hline
3 & dec & $B-1 \to B$ \\
\hline
4 & inc & $B+1 \to B$ \\
\hline
5 & neg & $-B \to B$ \\
\hline
6 & s.lt & Sets ``compare'' if $A < B$ (signed) \\
\hline
7 & u.lt & Sets ``compare'' if $A < B$ (unsigned) \\
\hline
\end{tabular}


\section{Logic}

The table below summarizes the math functions.

\begin{tabular}{|c||l|l|}
\hline
Encoding (PPPP) & Instruction & Behavior \\
\hline
\hline
0 & not & $\widehat B \to B$ \\
\hline
1 & nor & $\widehat{A+B} \to B$ \\
\hline
2 & and.n2 & $\widehat B A \to B$ \\
\hline
3 & clr & $0 \to B$ \\
\hline
4 & nand & $\widehat{BA} \to B$ \\
\hline
5 & copy.n1 & $\widehat A \to B$ \\
\hline
6 & xor & $B \oplus A \to B$ \\
\hline
7 & and.n1 & $B \widehat A \to B$ \\
\hline
8 & or.n2 & $\widehat B + A \to B$ \\
\hline
9 & xnor & $\widehat{B \oplus A} \to B$ \\
\hline
a & copy & $A \to B$ \\
\hline
b & and & $BA \to B$ \\
\hline
c & set & $1 \to B$ \\
\hline
d & or.n1 & $B + \widehat A \to B$ \\
\hline
e & or & $B + A \to B$ \\
\hline
f & tst & Sets the ``compare'' bit if $B$ is zero \\
\hline
\end{tabular}


\section{Immediate math and logic}

The table below summarizes the immediate-argument math and logic operations.

\begin{tabular}{|c||l|l|}
\hline
Encoding (PPP) & Instruction & Behavior \\
\hline
\hline
0 & iadd & $N+B \to B$ (arithmetic $+$) \\
\hline
1 & insub & $B-N \to B$ \\
\hline
2 & ior & $N+B \to B$ (logical $+$) \\
\hline
3 & iand & $NB \to B$ (logical ``and'') \\
\hline
\end{tabular}

\end{document}
